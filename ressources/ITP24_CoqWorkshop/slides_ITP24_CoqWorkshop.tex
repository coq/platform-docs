
\documentclass[10pt]{beamer}
\usepackage{import}




\subimport{./}{header_slides.tex}
\subimport{./}{metropolis_theme.tex}

\usepackage{csquotes}
\usepackage{biblatex}
\addbibresource{ref.bib}


\title{Coq Platform Docs: \\ Documenting Coq and its Platform}
\author{\textbf{Thomas Lamiaux}, ENS Paris-Saclay \\
Pierre Rousselin, LAGA, Université Sorbonne Paris Nord \\
Théo Zimmermann, LTCI, Télécom Paris, Polytechnic Institute of Paris }


\date{September 2024}

% latexmk -e '$max_repeat=6' slides_Abu_Dhabi.tex

\begin{document}

\begin{frame}
    \maketitle
\end{frame}


\section*{Coq Platform Docs}

\subsection{Introduction}

\begin{frame}{Why do we need such a documentation ?}
  \begin{tcbProp}{To keep in mind}
    \begin{enumerate}
      \item Having a proper, clean and accessible documentation is one of the
            keys to the success of software: \ul{to get and keep users}
      \item<2-> There are different forms of documentation:
      \begin{itemize}[label=$-$]
        \item Abstract and detailed documentation like the ref man
        \item Course-shaped documentation like Coq'Art or SF
        \item Action-oriented documentation like tutorials or how-to guides
      \end{itemize}
    \end{enumerate}
  \end{tcbProp}
  \onslide<3->
  \begin{tcbPbl}{A lack of action-oriented documentation }
    There are currently a lack of action-oriented documentation:
    \begin{enumerate}
      \item When it exits, it is often incomplete or outdated
      \item When it exits, it often spread out over many repo / blog
    \end{enumerate}
  \end{tcbPbl}
\end{frame}

\begin{frame}{Why is it important ?}
  \begin{tcbProp}{For users}
    \begin{enumerate}
      \item It enables user to learn and discover new features on their own and
            to debug themselves when stuck
      \item<2-> It enables to turn folklore into knowledge and make it accessible
      \item<3-> It enables to find specific knowledge easily
    \end{enumerate}
  \end{tcbProp}
  \onslide<4->
  \begin{tcbProp}{For experts}
    \begin{enumerate}
      \item Writing documentation forces us to do stuff right (no cheating)
      \item It makes our work easier to discover and more accessible
    \end{enumerate}
  \end{tcbProp}
\end{frame}

\subsection{The Project}

\begin{frame}{Presentation}
  \begin{tcbObj}{Coq Platform Docs}
    \begin{enumerate}
      \item<1-> Each features and plugins of Coq and its platform would have one
            or several \ul{interactive tutorials and how-to guides}
      \item<2-> It would be \ul{indexed on the Coq Platform} to be easy to use
      \item<3-> Possible to extend with explanations or learning games
    \end{enumerate}
  \end{tcbObj}
  \onslide<4->
  \begin{tcbObj}{Interactive Web interface}
    \begin{enumerate}
      \item<5-> It would be \ul{accessible online, interactively,} e.g., using JsCoq
      \item<6-> It should be \ul{part of Rocq's new website}
    \end{enumerate}
  \end{tcbObj}
\end{frame}

\begin{frame}{Advantages of such a documentation}
  \begin{tcbProp}{Advantages}
    \begin{enumerate}
      \item<1-> All the documentation should be \ul{accessible at the same place}
      \item<2-> It is possible to build \ul{modularly and collaboratively}
      \item<3-> Such a documentation should be \ul{horizontal:}
      \begin{itemize}[label=$-$]
        \item it easier to navigate and discover new features
        \item it easier to maintain as less dependencies
        \item it allows differentiate learning: \\
              Ltac2 for new users\;  /  \;Ltac2 for Ltac1's users
      \end{itemize}
      \item<4-> It will \ul{showcase all that is possible} with Coq
    \end{enumerate}
  \end{tcbProp}
\end{frame}

\begin{frame}{Tutorials vs How-to guides}
  \begin{tcbProp}{Tutorials}
    % Tutorials guide a user during learning in discovering specific aspects
    % of a feature like "Notations in Coq", by going through (simple)
    % predetermined examples, and introducing notions gradually.
    Tutorials guide users through learning new features:
    \begin{enumerate}
      \item It introduces notions gradually, step by step
      \item Going through predetermined nice examples
      \item It talks about \ul{one} specific feature
    \end{enumerate}
\end{tcbProp}
  \onslide<2>{
  \begin{tcbEx}{Examples Coq}
    \begin{itemize}[label=$-$]
      \item Searching for Lemma
      \item Requiring and Importing Files / Modules
      \item Setting up a Coq project
      \item Using Notations
    \end{itemize}
  \end{tcbEx}}
\end{frame}

\begin{frame}{Tutorials vs How-to guides}
  \begin{tcbProp}{Tutorials}
      % Tutorials guide a user during learning in discovering specific aspects
      % of a feature like "Notations in Coq", by going through (simple)
      % predetermined examples, and introducing notions gradually.
      Tutorials guide users through learning new features:
      \begin{enumerate}
        \item It introduces notions gradually, step by step
        \item Going through predetermined nice examples
        \item It talks about \ul{one} specific feature
      \end{enumerate}
  \end{tcbProp}
  \begin{tcbEx}{Examples Equations}
    \begin{itemize}[label=$-$]
      \item Introduction to Equations
      \item Equations and obligations
      \item Equations and well-founded recursion
      \item Equations and dependent programming
    \end{itemize}
  \end{tcbEx}
\end{frame}

\begin{frame}{Tutorials vs How-to guides}
  \begin{tcbProp}{How-to guides}
    How-to guides are meant to guide users through the resolution of real life
    problem:
    \begin{enumerate}
        \item Provide a step by step resolution procedure
        \item Accounting for real life complexity
        \item But without touring all that is possible
      \end{enumerate}
  \end{tcbProp}
  \onslide<2>{
  \begin{tcbEx}{Examples Coq}
    \begin{itemize}[label=$-$]
      \item How to search for a lemma
      \item How to set up a Coq Project
      \item How to use Equations to reason about intrinsic syntax
    \end{itemize}
  \end{tcbEx}}
\end{frame}

\begin{frame}{Example: Documenting Equations}
  \begin{tcbPbl}{The documentation was inadequate}
    \begin{enumerate}
      \item Many features: basics, dependent programming, wf recursion
      \item The documentation was old and not very complete
    \end{enumerate}
  \end{tcbPbl}
  \onslide<2->
  \begin{tcbProp}{A new documentation}
    \begin{enumerate}
      \item We wrote 3 tutorials out of 5 planned, ~500l code/~1000l text
      \item The documentation is now much clear, and easy to improve
      \item We wrote it modularly with help from the community
    \end{enumerate}
    \end{tcbProp}
  \onslide<3->
  \begin{tcbSol}{Many Improvements}
  \begin{enumerate}
    \item Found 2/3 bugs writing the documentation
    \item Improve the behavior of the main tactic
  \end{enumerate}
  \end{tcbSol}
\end{frame}

\subsection{Demonstration}

% 3 - 5 minutes


\subsection{Contributing}

% \subsection{Contributing to the documentation}

\begin{frame}{Documenting Project and Plugins}
  \begin{tcbProp}{Develop the documentation modularly}
    \begin{itemize}
      \item Each community should have a reviewer term and a project to manage
            its part of the documentation
    \end{itemize}
  \end{tcbProp}
  \onslide<2->
  \begin{tcbProp}{Basic Plan}
    \begin{enumerate}
      \item Set up a reviewer team and a project
      \item Understand what the documentation should look like
      \item Set up the associated ci
      \item Port / update / write the documentation
    \end{enumerate}
  \end{tcbProp}
\end{frame}

\begin{frame}{Contributing as a Coq's User}
  \begin{tcbProp}{Contributing by Reviewing}
    \begin{enumerate}
      \item Becoming an expert reviewer for a project or a general reviewer
      \item Review from time to time something you are an expert or not
    \end{enumerate}
  \end{tcbProp}
  \onslide<2->
  \begin{tcbProp}{Contributing by Writting}
    \begin{enumerate}
      \item Writing tutorials or how-to guides
      \item Help improve existing tutorials via P.R
    \end{enumerate}
  \end{tcbProp}
  \onslide<3->
  \begin{tcbProp}{Contributing by Sharing Knoweldge}
    \begin{enumerate}
      \item Share folklore that should be common knowledge
      \item Answer the questions of people writing tutorials
      \item Give feedbacks on existing tutorials and your needs
    \end{enumerate}
  \end{tcbProp}
\end{frame}


\begin{frame}{Contributing Technically}
  \begin{tcbProp}{Help with the Web interface}
    \begin{itemize}[label=$-$]
      \item Help setting up the web interface (set up Alectryon / Sphinx)
      \item Help setting up a nice automatic deployment
      \item Find a solution for JsCoq
    \end{itemize}
  \end{tcbProp}
  \onslide<2->
  \begin{tcbProp}{Help with the Repo}
    \begin{itemize}[label=$-$]
      \item Help setting up a nice ci (platform / dev etc..)
      \item Help writing contribution guidelines
      \item etc...
    \end{itemize}
  \end{tcbProp}
\end{frame}

\subsection{Conclusion}

% \begin{tcbProp}{Sum up}

% \end{tcbProp}
\begin{frame}[fragile]{Conclusion}
  \begin{tcbObj}{The Project}
    \begin{enumerate}
      \item Each features and plugins of Coq and its platform would have one
            or several \ul{interactive tutorials and how-to guides}
      \item Be accessible interactively online, and part of Coq's new website
    \end{enumerate}
  \end{tcbObj}
  \onslide<2->
  \begin{tcbObj}{Contributing}
    \begin{enumerate}
      \item We are open to contribution: there is plenty to do
      \item Everyone can contribute: you don't need to be a Coq developer
    \end{enumerate}
  \end{tcbObj}
  \onslide<3->
  \begin{tcbProp}{Ressources to check out}
    \begin{itemize}[label=$-$]
      \item Check out the \textcolor{blue}{\href{https://coq.inria.fr/platform-docs/}{prototype website and the first tutorials}}
      \item Come talk to us and give feedback on the \textcolor{blue}{\href{https://www.theozimmermann.net/platform-docs/}{zulip stream}}
      \item Check out the \textcolor{blue}{\href{https://github.com/coq/platform-docs}{repo}} and the \textcolor{blue}{\href{https://github.com/coq/ceps/pull/91}{CEP}}
    \end{itemize}
  \end{tcbProp}
\end{frame}


\end{document}