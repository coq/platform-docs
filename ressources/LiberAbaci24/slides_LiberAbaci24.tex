
\documentclass[10pt]{beamer}
\usepackage{import}




\subimport{./}{header_slides.tex}
\subimport{./}{metropolis_theme.tex}

\usepackage{csquotes}
\usepackage{biblatex}
\addbibresource{ref.bib}


\title{Coq Platform Docs: \\ Documenting Coq and its Platform}
\author{\textbf{Thomas Lamiaux}, ENS Paris-Saclay \\
Pierre Rousselin, LAGA, Université Sorbonne Paris Nord \\
Théo Zimmermann, LTCI, Télécom Paris, Polytechnic Institute of Paris }


\date{October 2024}

% latexmk -e '$max_repeat=6' slides_Abu_Dhabi.tex

\begin{document}

\begin{frame}
    \maketitle
\end{frame}


\section*{Coq Platform Docs}

\subsection{Why does Action-oriented Documentation Matter?}

\begin{frame}{Introduction}
  \begin{tcbProp}{To keep in mind}
    \begin{enumerate}
      \item Having a proper, clean and accessible documentation is one of the
            keys to the success of software: \ul{to get and keep users}
      \item<2-> There are different forms of documentation:
      \begin{itemize}[label=$-$]
        \item Abstract and detailed documentation like the ref man
        \item Course-shaped documentation like Coq'Art or SF
        \item Action oriented documentation like tutorials or how-to guides
      \end{itemize}
    \end{enumerate}
  \end{tcbProp}
  \onslide<3->
  \begin{tcbPbl}{A lack of action-oriented documentation }
    \begin{enumerate}
      \item When it exists, it is often incomplete or outdated
      \item When it exists, it is often spread out over many repos / blogs
    \end{enumerate}
  \end{tcbPbl}
\end{frame}

\begin{frame}{What is Action Oriented Documentation ?}
  \vspace*{-4pt}
  \begin{tcbProp}{Tutorials}
    % Tutorials guide a user during learning in discovering specific aspects
    % of a feature like "Notations in Coq", by going through (simple)
    % predetermined examples, and introducing notions gradually.
    Tutorials help users to learn a new feature, step by step, introducing
    notions gradually through predetermined examples:
    \begin{itemize}[label=$\hookrightarrow$]
      \item<2-> Searching for Lemma, Combining Tactics, etc...
    \end{itemize}
  \end{tcbProp}
  \vspace*{-4pt}
  \onslide<3->
  \begin{tcbProp}{How-to guides}
    How-to guides provide users with step by step resolution procedures for real
    life problems but without touring all that is possible:
    \begin{itemize}[label=$\hookrightarrow$]
      \item<4-> How to search for a lemma
      \item<5-> How to set up a Coq Project
    \end{itemize}
  \end{tcbProp}
  \vspace*{-4pt}
  \onslide<6->
  \begin{tcbSol}{To Remember}
    \begin{enumerate}
      \item Tutorials and how-guides are different and complementary
      \item They are both about practical knowledge and achieving goals
    \end{enumerate}
  \end{tcbSol}
\end{frame}

% TO WORK ON
\begin{frame}{Why is it important for Learning}
  % \begin{tcbPbl}{Have you noticed ?}
  %   \begin{itemize}
  %     \item That people have a hard time learning
  %     \item That people often tend to learn through experts
  %   \end{itemize}
  % \end{tcbPbl}
  % \begin{tcbProp}{}
  %   \begin{enumerate}
  %     \item Tutorials enables users to discover and learn new features on their own
  %     \item Tutorials enables users to become more experts on complex features
  %     \item How to guides enables users to step from learning to doing
  %     \item How-to guides enables users and to confront themselves
  %           to real life examples when working on projects
  %   \end{enumerate}
  % \end{tcbProp}
\end{frame}

\begin{frame}{It is important for Teaching}
  \begin{tcbProp}{When Teaching}
    \begin{enumerate}
      \item<1-> A lot about Coq is pratical knowledge and can be subtle
      \item<2-> It is not possible to be behind every student helping them
      \item<3-> You cannot teach everything in a 24h/48h class:
      \onslide<4->{
      \begin{itemize}[label=$-$]
        \item There are too many features, or too many aspects to discuss
        \item Some features are too advanced, or too technical to discuss
      \end{itemize}}
    \end{enumerate}
  \end{tcbProp}
  \onslide<5->
  \begin{tcbSol}{Action-oriented Documentation}
    % Action-oriented documentation enables to continue the class outside of class
    \begin{enumerate}
      \item<5-> Tutorials enable students to check and expand their knowledge
            through learning content and examples
      \item<6-> How-to guides provide students with practical examples helping them
            to debug themselves when working on projects
      \item<7-> Action-oriented documentation enables students to continue
            learning after class, and to get towards autonomy in Coq
    \end{enumerate}
  \end{tcbSol}
\end{frame}

\begin{frame}{It is important for our Community}
\begin{tcbProp}{For users}
  \begin{enumerate}
    \item It enables users to learn and discover new features on their own,
          and to debug themselves when they are stuck
    \item<2-> It enables users to find specific practical knowledge
    \item<3-> It enables us to turn folklore into knowledge, and to make it
          accessible to users that would not know it otherwise
  \end{enumerate}
\end{tcbProp}
\onslide<4->
\begin{tcbProp}{For experts}
  \begin{enumerate}
    \item<4-> It makes our work easier to discover and more accessible
    \item<5-> It helps getting new users, and getting them more experts
    \item<6-> It enables us to put forward and promote good practices
    \item<7-> Writing documentation forces us to do stuff right (no cheating)
    % \item<8-> Experts have features they want to learn too
  \end{enumerate}
\end{tcbProp}
\end{frame}

\subsection{The Project}

\begin{frame}{Presentation}
  \begin{tcbObj}{Coq Platform Docs}
    \begin{enumerate}
      \item<1-> Each feature and plugin of Coq and its platform would have one
            or several \ul{interactive tutorials and how-to guides}
      \item<2-> It would be \ul{indexed on the Coq Platform} to be easy to use
      \item<3-> Possible to extend with explanations or learning games
    \end{enumerate}
  \end{tcbObj}
  \onslide<4->
  \begin{tcbObj}{Interactive Web interface}
    \begin{enumerate}
      \item<5-> It would be \ul{accessible online, interactively,} e.g., using JsCoq
      \item<6-> It should be \ul{part of Rocq's new website}
    \end{enumerate}
  \end{tcbObj}
\end{frame}

\begin{frame}{Advantages of such a project}
  \begin{tcbSol}{Advantages for users}
    \begin{enumerate}
      \item<1-> Coq is an interactive and practical software, it \ul{needs} an
            interactive and practical documentation
      \item<2-> All the documentation should be \ul{accessible at the same place}
      \item<3-> It should be horizontal making it \ul{easier to discover} new features,
      \item<4-> It should be horizontal allowing \ul{differentiated learning}:
      \begin{itemize}[label=$\hookrightarrow$]
        \item Ltac2 for new users\;  v.s.  \;Ltac2 for Ltac1's users
      \end{itemize}
    \end{enumerate}
  \end{tcbSol}
  \onslide<5->
  \begin{tcbSol}{Advantages for the community}
    \begin{enumerate}
      \item<5-> It is possible to build \ul{modularly and collaboratively}
      \item<6-> It enables to \ul{centralise efforts} in a \ul{lasting way}
      \item<7-> Being horizontal, it should be \ul{easier to maintain}
      % \item<8-> It will \ul{showcase all that is possible} with Coq
    \end{enumerate}
  \end{tcbSol}
\end{frame}

\begin{frame}{Example: Documenting Equations}
  \onslide<1->
  \vspace*{-3pt}
  \begin{tcbPbl}{The documentation was inadequate}
    \begin{enumerate}
      \item Many features: basics, dependent matching, wf recursion, ...
      \item Despite many examples, the doc was not very fitted nor complete
    \end{enumerate}
  \end{tcbPbl}
  \onslide<2->
  \vspace*{-3pt}
   \begin{tcbProp}{A new documentation}
    \begin{enumerate}
      \item We wrote 4 tutorials out of 6 planned, ~600l code/~1200l text
      \item It is now much clear, and easy to improve or extend
      \item We wrote it modularly with help from the community
    \end{enumerate}
  \end{tcbProp}
  \only<2>{
  \vspace*{-15pt}
  \begin{figure}[H] \hspace*{-0.9cm}
    \includegraphics[scale=0.35]{equations_tuto.jpg}
  \end{figure}}
  \only<3>{
  \vspace*{-3pt}
  \begin{tcbSol}{Many Improvements}
  \begin{enumerate}
    \item We found 2/3 bugs writing the documentation
    \item We improved the behavior of the main tactic
  \end{enumerate}
  \end{tcbSol}}
\end{frame}

\subsection{Demonstration}

% 3 - 5 minutes

\begin{frame}
  \begin{center}
    \large
    Search for \textbf{Coq Platform Docs}, or go on: \\
    \textcolor{blue}{\url{https://coq.inria.fr/platform-docs/}}
  \end{center}
\end{frame}


\subsection{Contributing}

% \subsection{Contributing to the documentation}

\begin{frame}{Contributing}
  \begin{tcbObj}{Everyone can help}
    \begin{enumerate}
      \item<1-> \ul{Everyone can contribute} whatever their expertise !
      \item<2-> \ul{Both} experts and casual users' opinions are valuable !
    \end{enumerate}
  \end{tcbObj}
  \onslide<3->
  \begin{tcbProp}{Contributing}
    There are \ul{many ways} to contribute:
    \begin{enumerate}
      \item<3-> Give feedback on existing tutos / how-tos, or wishes on
      \textcolor{blue}{\href{https://www.theozimmermann.net/platform-docs/}{zulip}}
      \item<4-> Answer questions, and share knowledge and folklore on
      \textcolor{blue}{\href{https://www.theozimmermann.net/platform-docs/}{zulip}}
      \item<5-> Write tutorials and how-to guides, or write PR to improve them
      \item<6-> Help reviewing tutorials and how-to guides
      \item<7-> Help setting up the repo, or the web-interface
    \end{enumerate}
  \end{tcbProp}
\end{frame}

\begin{frame}{Documenting Project and Plugins}
  \begin{tcbProp}{Develop the documentation modularly}
    For each projet or plugins, we would like to:
    \begin{enumerate}
      \item<2-> Set up a reviewer team with dev / maintainer / advanced users
      \item<2-> Set up a project and open issues so that people can contribute
    \end{enumerate}
  \end{tcbProp}
  \onslide<3->
  \begin{tcbProp}{Being a reviewer}
    The job of the reviewers is not to write documentation, it is to:
    \begin{enumerate}
      \item<4-> Identify the needs of documentation, and create issues for it
      \item<5-> Ensure that the documentation stays consistent through PR
      \item<6-> Review contributions when some are created
    \end{enumerate}
  \end{tcbProp}
\end{frame}

\subsection{Conclusion}

\begin{frame}[fragile]{Conclusion}
  \begin{tcbObj}{The Project}
    \begin{enumerate}[leftmargin=10pt]
      \item Each feature and plugin of Coq and its platform would have one
            or several \ul{interactive tutorials and how-to guides}
      \item Be accessible \ul{interactively online}, and part of \ul{Rocq's new website}
    \end{enumerate}
  \end{tcbObj}
  \onslide<2->
  \begin{tcbObj}{Contributing}
    \begin{enumerate}[leftmargin=10pt]
      \item Everyone can contribute: you don't need to be a Coq developer
      \item Everyone can have a valuable contribution
    \end{enumerate}
  \end{tcbObj}
  \onslide<3->
  \begin{tcbProp}{Ressources to check out}
    \begin{itemize}[label=$-$,leftmargin=10pt]
      \item Check out the \textcolor{blue}{\href{https://coq.inria.fr/platform-docs/}{prototype website and the first tutorials}}
      \item Come talk to us and give feedback on the \textcolor{blue}{\href{https://www.theozimmermann.net/platform-docs/}{zulip stream}}
      \item Check out the \textcolor{blue}{\href{https://github.com/coq/platform-docs}{repo}} and the \textcolor{blue}{\href{https://github.com/coq/ceps/pull/91}{CEP}}
    \end{itemize}
  \end{tcbProp}
\end{frame}


\end{document}